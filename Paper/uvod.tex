Problem pronalaženja homolognih poklapanja unutar dva niza je problem koji je značajan za mnoga područja moderne znanosti, a pogotovo je važan u području bioinformatike. Zbog brzog razvoja različitih metoda sekvenciranja genoma, koje postaju sve jeftinije i pristupačnije, u bioinformatici se javila potreba za razvojem algoritama koji mogu efikasno rješavati problem pronalaženja poravnanja homolognih preklapanja unutar genoma. Sekvenciranjem genoma dobivamo sekvence koje je moguće jednostavno prikazati u računalu, jer se svaka aminokiselina može preslikati u jedan znak. Drugim riječima, jedno očitanje molekule DNK se može preslikati u niz znakova. Time se problem pronalaženja sličnih djelova sekvenci dobivenih sekvenciranjem DNK pretvara u problem pronalaženja preklapanja unutar dva niza znakova. Također, vidljivo je da se ovakvi algoritmi mogu primjenjivati na bilo koja dva niza znakova, no u ovom radu fokusirat ćemo se na njihovu primjenu u bioinformatici i preklapanju sekvenci DNK.

Iako se na prvi pogled taj problem može činiti jednostvnim, do problema dolazimo kada u obzir uzmemo moguće duljine sekvecni koje mogu imati od nekoliko desetaka proteinskih baza, pa i do čak nekoliko stotina tisuća proteinskih baza, a da ukupni broj baza u svim sekvencama može biti i do nekoliko desetaka milijardi. Ako bi problem probali riješiti nekim jednostavnim algoritmom koji bi, ako uspoređujemo dvije sekvence od kojih jedna ima duljinu \textit{m}, a druga duljinu \textit{n}, njegova složenost bila bi \textit{O}(\textit{nm}), vidljivo je da bi vrijeme potrebno za riješavanje bilo preveliko. Zbog toga dolazimo do potrebe za efikasnijim algoritmima poravnanja, koje možemo podijeliti na heurističke i analitičke. Heuristički algoritmi se baziraju na procjeni poravnanja korištenjem različitih heurističkih funkcija te su samim time nešto neprecizniji od analitičkih, ali tu nepreciznost nadoknađuju u brzini pronalaženja rješenja. S druge strane, analitički algoritmi uvode neka ograničenja na rješenja, te si time smanjuju ukupni prostor rješenja koje moraju pretražiti. Algoritam Landau-Vishkin-Nussinov spada u skupinu analitičkih algoritama te će dalje u radu biti razmatrani samo analitički algoritmi. 

U suštini, problem traženja poravnanja između dva niza svodi se na traženje najmanjeg broja razlika između ta dva niza. Drugi naziv za broj razlika između dva niza je udaljenost(eng. edit distance). Udaljenost dva niza možemo promatrati i kao broj transforacija koje je potrebno napraviti da bi se prvi niz pretvorio u drugi niz. Na primjer, najmanja udaljenost nizova "Hello" i "World!" je 5 jer sve znakove osim znaka l moramo promjeniti kako bi iz niza "Hello" dobili niz "World!". 